\documentclass[conference]{IEEEtran}
\IEEEoverridecommandlockouts
% The preceding line is only needed to identify funding in the first footnote. If that is unneeded, please comment it out.
\usepackage{cite}
\usepackage{amsmath,amssymb,amsfonts}
\usepackage{algorithmic}
\usepackage{graphicx}
\usepackage{textcomp}
\usepackage{xcolor}
\def\BibTeX{{\rm B\kern-.05em{\sc i\kern-.025em b}\kern-.08em
    T\kern-.1667em\lower.7ex\hbox{E}\kern-.125emX}}
\begin{document}

\title{Estatísticas de preços de automóveis de passeio - AutoStats\\
\thanks{}
}

\author{\IEEEauthorblockN{Artur P. Carneiro}
\IEEEauthorblockA{\textit{Dept. De Inteligência Artificial Aplicada à Robótica} \\
\textit{FEI - Fundação Educacional Inaciana}\\
São Bernardo do Campo, Brasil \\
acarneiro@fei.edu.br}
}

\maketitle

\begin{abstract}
Esse artigo sintetiza o protótipo de uma solução para pesquisa estatística sobre os valores dos automóveis de passeio (novos e usados) no Brasil.

Esse protótipo é a resposta à um trabalho de conclusão do curso da disciplina de Ciência de Dados, ministrada para os alunos de pós-graduação dos cursos de Mestrado e Doutorado da FEI - Fundação Educacional Inaciana.
\end{abstract}

\begin{IEEEkeywords}
ciência de dados, automóveis de passeio, valor, estatística
\end{IEEEkeywords}

\section{Introdução}
Explicar ou equacionar a evolução dos preços de veículos automotores de passeio (também chamados de carros) no mercado brasileiro, usados ou novos, tem se mostrado um verdadeiro desafio para o comprador. Semelhante desafio é planejar o momento de venda/troca do veículo ou estimar o valor de compra/venda deste. Os fatores que influenciam nos valores dos carros novos vão desde questões produtivas (como a queda da oferta de componentes eletrônicos durante a pandemia do Corona Vírus) até questões econômicas (como a entrada de novos \textit{players} `agressivos', com preços menores para carros mais tecnológicos, ou a redução temporária de impostos pelo governo federal), passando por ações protecionistas dos fabricantes de veículos que optam por parar de produzir e manter os estoques cheios à baixar a margem de lucro por unidade e ganhar na escala, aumentando a produção e esvaziando os estoques. Enquanto isso, o valor dos carros usados é impactado, principalmente, pela capacidade de compra do carro novo pela população, fazendo os preços dos carros usados aumentarem quando a quantidade de novos vendidos diminui e vice-versa. 

Toda essa dinâmica complexa impacta na tomada de decisão por parte do comprador/vendedor, que deve decidir, além do valor inicial da compra (à vista ou financiada) do veículo, questões como custos de seguro, manutenção, documentação e desvalorização do veículo até o momento de "revenda" futuro.

Auxiliar o comprador/vendedor nesta tarefa de análise e tomada de decisão é a motivação da solução proposta pelo protótipo aqui apresentado.

A solução aqui descrita propõe coletar dados sobre a evolução de preços dos veículos em outros sistemas independentes (Fundação Instituto de Pesquisas Econômicas - Fipe, OLX e WebMortors) e indicadores econômicos (Índice Nacional de Preços ao Consumidor Amplo - IPCA, Índice Nacional de Preços ao Consumidor - INPC, Índice de Preços ao Consumidor - IPC-Fipe e o Índice Geral Preços - IGP-M ), concentrá-los e organizá-los para apresentar ao usuário gráficos e informações que sirvam de insumos para a análise de tomada de decisão quanto à compra/venda de um veículo.

O Desenvolvimento é orientado às etapas do ciclo dos dados:

\begin{itemize}
	\item Geração
	\item Coleta
	\item Processamento
	\item Armazenamento
	\item Gerenciamento
	\item Análise
	\item Visualização
	\item Interpretação
\end{itemize}

Por tratar-se de um protótipo que visa demonstrar a viabilidade da solução, os dados são limitados à posteriores a janeiro de 2013 e as informações produzidas se restringem a modelos da marca Volkswagen onde, dado um modelo de veículo específico, são gerados gráficos comparativos da evolução do valor de em relação aos índices de preço, percentual de desvalorização mensal e evolução do preço deste modelo novo (zero quilômetro).

\section{Geração}

\subsection{Preço dos veículos de passeio }

Como fonte de informação sobre preços de veículos de passeio novos e usados, utiliza-se três origens distintas. A primeira refere-se à origem oficial do preço médio de veículos de passeio com valores  reconhecidos por instituições financeiras como a referência de valor dos veículos, utilizados em contratos e decisões judiciais. Esses valores são disponibilizados em consulta gratuita pela {Fipe}, na página https://veiculos.fipe.org.br/.  

As outras fontes de informações são sites de vendas (WebMotors e OLX), que concentram anúncios de diversos lojistas e/ou pessoas físicas interessadas em vender seus veículos. O WebMotors é um \textit{"marketplace"} exclusivamente de veículos novos ou usados. O OLX, por sua vez, possui uma variedade de produtos, mas é muito utilizado por pessoas interessadas em vender carros usados também.

\subsection{Indicadores Econômicos}

O portal {Debit} disponibiliza em  `https://debit.com.br/tabelas/ipca-indice-nacional-de-precos-ao-consumidor-amplo.php' uma tabela com a variação mensal do IPCA desde 1980. Esses valores são fornecidos pelo {IBGE}, mensalmente.

Esta mesma solução disponibiliza a tabela do Índice Geral de Preços Médios (IGP-M), produzido pela Fundação Getúlio Vargas - {FGV}, desde 1989 em 'https://debit.com.br/tabelas/tabela-completa.php?indice=igpm´.

O portal da Fipe disponibiliza as taxas mensais do Índice de Preços ao Consumidor - IPC, desde 2011, em 'https://www.fipe.org.br/pt-br/indices/ipc/\#indice-mensal´.


\section{Coleta}

\subsection{Preço médio de veículos de passeio - Fipe}

A solução da Fipe utiliza de serviços para carregar as informações dinamicamente na página (o que chamamos de AJAX) e realizar as consultas (serviços disponibilizados em 'https://veiculos.fipe.org.br/api/veiculos´). Esses serviços não são controlados ou autenticados. Isso possibilita executar um processo de captura dos dados consumindo diretamente os serviços. São eles:

\begin{itemize}
	\item 'ConsultarTabelaDeReferencia´: retorna uma lista de referências (mês/ano) possíveis. Compreende ao mês e o ano da coleta dos preços médios dos veículos.
	\item 'ConsultarMarcas´: retorna as marcas presentes uma dada referência.
	\item 'ConsultarModelos´: retorna os modelos existentes nos resultados, dado a marca e referência.
	\item 'ConsultarAnoModelo´ : retorna os anos de fabricação existentes, dado o modelo, marca e referência.
	\item 'ConsultarValorComTodosParametros´ : retorna o valor de um veículo dado um ano de fabricação, modelo, marca e referência. 
\end{itemize}

O protótipo, então, realiza a coleta dos dados de preços por um \textit{'script´ Python}, total ou parcial, atualizando esses dados no banco de dados utilizado na aplicação.


\subsection{Preço médio "Real" de venda em estabelecimentos comerciais}

Os \textit{marketplaces} WebMotors e OLX foram desenvolvidos de maneira à não permitir que requisições não vindas de seus próprios \textit{sites} acessem os serviços "AJAX". Desta forma, não é possível utilizar a mesma estratégia de coleta utilizada no portal da Fipe.

Nestes casos utiliza-se da estratégia chamada \textit{'web crawler´} onde um \textit{script} Python realiza uma 'navegação sistêmica´ aos portais e captura as informações geradas, passando-se por um usuário humano. 

Para OLX, faz-se uma requisição ao endereço:

https://www.olx.com.br/brasil?q=\{FABRICANTE\} \{MODELO\} \{ANO\}

Para o WebMotors, navega-se em:

https://www.webmotors.com.br/carros?tipoveiculo=carros\& anoate=\{ANO\}\&anode=\{ANO\}\&marca1=\{FABRICANTE\}\& modelo1=\{MODELO\}\&versao1=\{VERSÃO DO MODELO\}

Por definição, os anúncios são exibidos em ordem de "relevância" (conceito definido pela empresa - WebMotors ou OLX). Uma vez acessadas essas URLs, captura-se os campos HTML reverentes aos preços de até 20 anúncios na ordem de exibição.

Computa-se então, 3 valores (Mínimo, Médio e Máximo) para cada um dos portais, para o veículo correspondente.

\subsection{Indicadores Econômicos}

A coleta dos indicadores econômicos é realizada por um arquivo '.csv´ com colunas: Ano, Mês, IPC Geral, IGPM e IPCA.
 
Um script Python processa o arquivo e armazena no banco de dados.

O arquivo ".csv" é produzidos manualmente, em planilhas eletrônicas, com base nas informações acessadas nas páginas citadas como fonte de dados.

\section{Processamento e Armazenamento}

Os dados são processados e estruturados em um banco de dados relacional. Os próprios \textit{scripts} de carga realizam a organização dos dados e a distribuição nas tabelas.


\begin{figure}[htbp]
	\centerline{\includegraphics[width=250pt]{assets/database.png}}
	\caption{Modelo Entidade Relacionamento.}
	\label{fig1}
\end{figure}

Tabelas auxiliares para otimizar a geração dos gráficos são produzidas por \textit{scripts} periódicos com os dados de variação percentual do preço entre por referência (mês/ano) e modelo (código Fipe).

\section{Gerenciamento e Análise}

Para veículos com mais de 3 anos de uso, a solução realiza a previsão dos valores para 6 meses futuros, fornecendo ao usuário uma perspectiva para compra e venda.

Essa previsão é realizada com uma regressão utilizando um algoritmo KNN (\textit{K-Nearest Neighbors}) com k=3, atingindo um coeficiente de determinação (R\textsuperscript{2}) igual a 98,83%.


\begin{figure}[htbp]
	\centerline{\includegraphics[width=250pt]{assets/previsao.png}}
	\caption{Gráfico da predição de 6 meses do preço.}
	\label{fig2}
\end{figure}


\section{Visualização e Interpretação}

Selecionado uma marca, um modelo e o ano de fabricação do veículo, a solução disponibiliza um \textit{`dashboard'} com o valor atual do veículo (segundo a Fipe), os valores mínimos, médios e máximos segundo o WebMotors e OLX e os gráficos:

\begin{itemize}
	\item série temporal do valor do ano de fabricação até o mês atual;
	\item série temporal do valor `zero quilômetro' do ano de fabricação até o mês em que era disponível o modelo novo;
	\item série temporal demonstrando a variação percentual do valor do ano de fabricação até o mês atual junto com as curvas dos índices econômicos no mesmo período;
	\item série temporal da variação percentual do valor do veículo `zero quilômetro' junto com as curvas dos índices econômicos no mesmo período.
\end{itemize}

A seguir exemplos destes gráficos para o veículo Modelo de código Fipe 005481-0 - `Fox Connect 1.6 Flex 8V 5p'.

\begin{figure}[htbp]
	\centerline{\includegraphics[width=250pt]{assets/precoUsado.png}}
	\caption{Preço do veículo usado.}
	\label{fig3}
\end{figure}


\begin{figure}[htbp]
	\centerline{\includegraphics[width=250pt]{assets/precoZero.png}}
	\caption{Preço do veículo novo.}
	\label{fig4}
\end{figure}


\begin{figure}[htbp]
	\centerline{\includegraphics[width=250pt]{assets/varZero.png}}
	\caption{Variação do preço do veículo novo em comparação com índices financeiros.}
	\label{fig5}
\end{figure}


\begin{figure}[htbp]
	\centerline{\includegraphics[width=250pt]{assets/varUsado.png}}
	\caption{Variação do preço do veículo usado em comparação com índices financeiros.}
	\label{fig6}
\end{figure}

É disponibilizado no \textit{`dashboard'}, também, os valores previstos para os próximos 6 meses pelo algoritmo.

\begin{thebibliography}{00}
	\bibitem{Fipe} Fipe - Fundação Instituto de Pesquisas Econômicas. https://www.fipe.org.br.
	\bibitem{Debit} Debit - https://debit.com.br.
	\bibitem{FGV} Fundação Getúlio Vargas - Índice Geral de Preços - Mercado https://portal.fgv.br/noticias/igp-m-resultados-2023
	\bibitem{IBGE} IBGE- Instituto Brasileiro de Geografia e Estatística - IPCA - Índice Nacional de Preços ao Consumidor Amplo https://www.ibge.gov.br/estatisticas/economicas/precos-e-custos/9256-indice-nacional-de-precos-ao-consumidor-amplo.html
\end{thebibliography}
\end{document}
